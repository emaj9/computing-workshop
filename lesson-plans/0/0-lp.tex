\documentclass[12pt]{article}
\usepackage[margin=2cm]{geometry}
\usepackage{hyperref}
\usepackage{dialogue}
\usepackage{polyglossia}
\title{0 - Introductory Lesson}
\author{Eric Mayhew}
\date{}
\newcommand{\cwurl}{https://www.computing-workshop.com/lesson/pdfs/}

\begin{document}
\renewcommand{\abstractname}{\vspace{-\baselineskip}}
%%%This will remove the "Abstract" title from abstract
\maketitle
\begin{abstract}
This lesson marks the first step in our journey to making a website. Besides
  getting to know one another,
  partipants will disassemble a computer to get a hands-on, bird's eye view of
  hardware. Then by discussing abstractions in computer science, this lesson
  will cover software
  and its relationship to hardware. This provides participants with a general
  foundation about computers that will be expanded upon over the course
  of the workshop.
\end{abstract}
\section*{Overview}
\begin{description}
  \item [Lesson Objectives] 
    ~

   By the end of the lesson, students will be able to:
  \begin{itemize}

    \item Identify and describe the function of common computer components,

    \item Articulate a working definition of abstraction (in the context of
      computer science).

  \end{itemize}
  \item [Materials]~

Here's what you'll need to run this lesson:
  \begin{itemize}
    \item
      lesson slides;

      \url{www.computing-workshop.com/lessons/pdfs/0-slides.pdf}

    \item
      ice breaker activity, 1 copy per group; 

      \url{\cwurl 0-icebreaker.pdf}

    \item
      name tags for participants;

    \item
      blank paper and colouring supplies;

    \item
      computer disassembly worksheet, 1 copy per group;

      \url{www.computing-workshop.com/lessons/pdfs/0-cpu-disassembly.pdf}

    \item
      computers for disassembly (and tools for disassembly); ideally 1
      computer;
      for 4 students.

    \item
      abstraction activity worksheet, 1 copy per group; and

      \url{www.computing-workshop.com/lessons/pdfs/0-abstraction-activity.pdf}

    \item
      abstraction notes worksheet, 1 per person.

      \url{www.computing-workshop.com/lessons/pdfs/0-abstraction-notes.pdf}

  \end{itemize}
\end{description}
  \section*{Instructional Sequence}
  \begin{itemize}
    \item[10 minutes] Introduce course website, explain navigation of the
      website, introduce syllabus, introduce facilitators
    \item[25 minutes] Assign students to groups of four, getting them them to complete the icebreaker activity worksheet. They will have 15 minutes to prepare and 2
      minutes per group to present their groups. At the end of the session be
      sure to preserve the group banners.
    \item[20 minutes] Present groups with the computer they will disassembly,
      the tools to disassemble, and the worksheet they are to fill out.
      Circulate amongst the groups to ensure they're on the right track. If
      they need help finding online resources, direct them to the resources tab of the
      workshop website. 
    \item[10 minutes] Get the partipants to describe the function of each
      computer component they removed from their computer; each group should
      present at least 1 component.
    \item[10 minutes] Watch the videos in the slides.
    \item[20 minutes] Use the anology of drawing to first introduce students to
      the concept of abstraction. Ask one student to come up to the board and ask them to
      draw a square. Upon completing the square, ask the student how they know
      By showing the way we associate more complicate processes under one words
      like square, we are using abstraction. Here is an example as to how the
      conversation might unfold during a class.
      \begin{dialogue}
        \speak{Facilitator} To best understand abstraction, we are going to
        apply the concept of abstraction to drawing. Can a volunteer come up to
        the board to draw a square?
        \speak{Participant} I volunteer!

        \direct{Participant comes to the board and draws a square}

        \speak{Fac.} Thank you. Now, how do you know what a square is?
        \speak{Par.} Well I know a square is like a rectangle except with
        equally long sides.
        \speak{Fac.} This is true, and already we have an abstraction. Can
        someone identify where our friend used abstraction?
        \speak{Par. 2} They used the word "square" to describe a square as a
        rectangle with equally long sides. They replace the definition of the
        square with just the word square.
        \speak{Fac.} Yes, percisely. Square is an abstraction.
      \end{dialogue}
      ~
      Moving on, emphasis the 2 main uses of abstraction: easy of use and
      composability. To describe composability, continue using the drawing
      anology to describe how multiple processes can be built up into one
      abstraction.
      \begin{dialogue}
        \speak{Facilitator} Now that we have a working defintion of a square,
        let's suppose we really just want to draw a house. Assuming we already
        defined a triangle, let's say a house is a triangle on a square.

        \direct{Facilitator draws a square and triangle to looks like a house}

        \speak{Fac.} This here is another level of abstraction; square plus
        triangle equals house. We can continue
        to use abstractions to further develop our drawing. Let's assume a
        village is a group of houses close together.

        \direct{Facilitator draws a group of houses, all made from triangles
        built atop squares}

      \end{dialogue}
      Connect the drawing analogy back to the realm of
      computers, emphasizing how abstraction allows us to make powerful and
      efficient programs and commands.
      \begin{dialogue}
        \speak{Facilitator} We are able to created new abstractions by combining
        different abstractions. By combining abstractions like square,
        triangle, and house, we were able to come to villages. This same theory
        applies to not what makes programming so powerful but also can describe
        a way to categorize all the different levels of a computer, from
        Minecraft to circuits.
      \end{dialogue}
    \item[20 minutes] Distribute the abstraction worksheet and assign each group a level of abstraction for them to
      reserach and teach to the rest of the class.
    \item[5 minutes] Conclude class with review of what was covered and what
      will be covered next class.
  \end{itemize}
\end{document}
