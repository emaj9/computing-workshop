\documentclass[12pt]{article}
\usepackage[margin=2cm]{geometry}
\usepackage{hyperref}
\title{1 - First Foray into Software}
\author{Eric Mayhew \& Jacob Errington}
\date{}
\newcommand{\cwurl}{https://www.computing-workshop.com/lesson/pdfs/}

\begin{document}
\renewcommand{\abstractname}{\vspace{-\baselineskip}}
%%%This will remove the "Abstract" title from abstract
\maketitle
\begin{abstract}

\end{abstract}
\section*{Overview}
\begin{description}
  \item [Lesson Objectives] 
    ~

   By the end of the lesson, students will be able to:
  \begin{itemize}

    \item Identify common elements of a website.

    \item Describe the boot process of a computer.

    \item Articulate the roles of software, programs and operating
      system.

  \end{itemize}
  \item [Materials]~

Here's what you'll need to run this lesson:
  \begin{itemize}
    \item
      lesson slides;

      \url{\cwurl 1-slides.pdf}

    \item
      Graph paper for participants;

    \item
      Group review activity worksheet;

      \url{\cwurl 1-group-review.pdf}
  \end{itemize}
\end{description}
  \section*{Instructional Sequence}
  \begin{itemize}
    \item[5 minutes] Welcome group and present lesson agenda.
    \item[10 mins.] Present participants with the website activity, which
      invites participants to share the website they find most beautiful. After
      each group member has shared their site of choice, each group must select
      1 website to show to the rest of the group. 
    \item[5 mins.] Generate a group discussion
      around common elements that make a good website and common website
      layouts. This should culminate in identifying the header, nagivation,
      body, and footer of websites. Be sure to write the group discussion's
      findings on
      the board for participants to reference for the next activity.
    \item[10 mins.] Participants will now design the website they will create at
      the end of the workshop using colour supplies and graph paper. The slides
      contain
      prompts to get started. Make sure they
      keep this template handy as it will be used when partipants create their
      website.
    \item[10 mins.] In the next activity, one partipant will teach the boot
      process of a computer to
      their groupmates. Building off last week's activity disassembling a
      computer, the lesson will focus on what happens when you turn a computer on.
      All but one groupmember will observe the steps of
      the boot process by repeatedly restrating a member's computer and noting
      when certain events take place (as in noises, changes on screen, so on). In the
      meantime, the one removed member will go to another room with one of the
      two facilitators to learn through direct insturction the boot process.
      This video will be used to help explain the process: \url{https://www.youtube.com/watch?v=zyHoBzm5taw}.
    \item[10 mins.] When all partipants understand, they will be reunited with
      their groups and tasked with identifying the boot process steps presented
      on the following slide.
    \item[5 mins.] Facilitators will review the boot process with the groups,
      calling on groups to help explain different steps.
    \item[15 mins.] Moving on to operating systems (OSs), a crash course video
      will explain OSs and their function. Afterwards, facilitators will
      clarify the key differences of programs and OSs. 
    \item[10 mins.] Before beginning our introducion to Haskell, partipants
      will do a group activity to review concepts covered in the workshop this
      far. Facilitators will write a question on the front and back of 4 sheets
      of paper. The questions are:
      \begin{enumerate}
        \item Given an example of a responsibility of the operating system;
        \item Name one difference between an operating system and a program;
        \item What is an essential physical component of a computer (as in, a
          computer can't run without it)?
        \item Name one responsibility of hardware;
        \item What are the key features of a website?
        \item What are qualities that make a good website?
        \item Name the next step in the boot sequence of a computer;
        \item Name one level of abstraction, placing your answer in the right
          spot hierarchically.
      \end{enumerate}
      Each group will have 60 seconds to answer the question, then they will
      rotate the pages to the next group clockwise. Once all questions are
      answered, they will flip the pages over and answer the questions on the
      backside, repeating the process.
    \item[10 mins.] The rest of the class will introduce Haskell and
      programming languages. To tie in pre-existing knowledge of hardware,
      students will understand compilation to be moving down the latter of
      abstraction. Participants are now prepared to complete the
      \url{www.tryhaskell.org} by the next session.
  \end{itemize}
\end{document}
