\documentclass{beamer}

\usepackage{graphicx}
\usepackage{pgfplots}

\usefonttheme{serif}

\title{1 - Binary and logic}
\author{}
\date{}

\begin{document}

\frame{\titlepage}

\begin{frame}
  \frametitle{Agenda}
  \tableofcontents
\end{frame}

\section{Inside the computer: power supply unit}

\begin{frame}
  \frametitle{Power supply unit (PSU)}

  \begin{itemize}
    \item Elecrical power comes from the \emph{movement of electrons}.
    \item Electrons can move in different \emph{patterns}, each of which can be
      more convenient for different applications.
    \item The PSU \emph{transforms} the movement of electricity in your walls
      into a movement suitable for a computer.
  \end{itemize}

  % \visible<2>{
  %   \begin{tikzpicture}
  %     \begin{axis}[domain=0:10000,legend pos=outer north east]
  %       \addplot{sin(x)};
  %       \legend{AC}
  %     \end{axis}
  %   \end{tikzpicture}
  % }
\end{frame}

\section{Base $2$: binary}

\begin{frame}
  bases are cool
\end{frame}

\section{Circuits and logic gates}

\begin{frame}
  what's a function anyway?
\end{frame}

\section{Our first Haskell code}

\begin{frame}
  circuits are like pzzz bang boom
\end{frame}

\end{document}
