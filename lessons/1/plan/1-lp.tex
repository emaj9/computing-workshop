\documentclass[11pt]{article}
\usepackage[margin=2cm]{geometry}
\usepackage{hyperref}
\title{1 - Binary and logic}
\author{Eric Mayhew \& Jacob Errington}
\date{}
\newcommand{\cwurl}{https://www.computing-workshop.com/lesson/pdfs/}

\begin{document}
\maketitle

\section*{Overview}

\begin{description}
  \item[Lesson Objectives] ~

   By the end of the lesson, students will be able to:

  \begin{itemize}
    %\item Identify common elements of a website.
    \item Explain the function of the power supply unit in the computer.

    %\item Describe the boot process of a computer.
    %\item Represent and operate on data in binary.

    %\item Articulate the roles of software, programs and operating
    %  system.
    \item Build logic gates out of transistors.

    \item Use logic gates to implement binary functions in hardware.

    \item Name a few basic Haskell types and use an interactive Haskell prompt
      like GHCi.
  \end{itemize}

  \item [Materials]~

    To run this lesson, the following are needed:

    \begin{itemize}
      \item
        lesson slides;

        \url{\cwurl 1-slides.pdf}

      \item
        circuit diagrams and boolean logic worksheet;

        \url{\cwurl 1-ws-circuit.pdf}
    \end{itemize}
\end{description}

\section*{Instructional Sequence}

\begin{itemize}
  \item[5 mins.]
    Facilitators welcome the group and present lesson agenda.

  \item[20 mins.]
    Using the slides, facilitators will present the power supply unit, binary
    numbers, and logic gates (AND, OR, NOT, XOR).

  \item[15 mins.]
    Students will complete the logic gates worksheet.

  \item[10 mins.]
    The rest of the class will introduce Haskell and programming languages. To
    tie in pre-existing knowledge of hardware, students will understand
    compilation to be moving down the latter of abstraction. Participants are
    now prepared to complete the \url{www.tryhaskell.org} by the next session.
\end{itemize}

\end{document}
