\documentclass[11pt]{article}
\usepackage[margin=2cm]{geometry}
\usepackage{hyperref}
\title{1 - Binary and logic}
\author{Eric Mayhew \& Jacob Errington}
\date{}
\newcommand{\cwurl}{https://www.computing-workshop.com/}
\newcommand{\cwpdf}{\cwurl pdf/}

\begin{document}
\maketitle

\section*{Overview}

\begin{description}
  \item[Lesson Objectives] ~

   By the end of the lesson, students will be able to:

  \begin{itemize}
    \item Explain the function of the power supply unit in the computer.

    \item Build basic logic gates out of transistors, both theoretically and
      physically on a breadboard.

    \item Draw the logic gate diagram corresponding to a circuit on a
      breadboard.

    \item Name a few basic Haskell types and use an interactive Haskell prompt
      like GHCi.
  \end{itemize}

  \item [Materials]~

    To run this lesson, the following are needed:

    \begin{itemize}
      \item
        lesson slides;

        \url{\cwpdf 1-slides.pdf}

      \item
        logic gate review worksheet;

        \url{\cwpdf 1-ws-gate-review.pdf}

      \item
        circuit diagrams and boolean logic worksheet;

        \url{\cwpdf 1-ws-circuit.pdf}

      \item
        breadboard reverse engineering worksheet;

        \url{\cwpdf 1-ws-bb-reverse-engineering.pdf}

      \item
        one breadboard for each group and enough of the following electronic
        components:
        NPN transistors,
        100~$\Omega$ resistors,
        9~V batteries,
        5~V power regulators,
        jumper wires of various lengths or ribbon cables,
        and assorted LEDs.

    \end{itemize}
\end{description}

\section*{Instructional Sequence}

\begin{itemize}
  \item[5 mins.]
    Facilitators welcome the group and present lesson agenda.

  \item[20 mins.]
    Using the slides, facilitators will present the power supply unit, binary
    numbers, and logic gates (AND, OR, NOT, XOR).

  \item[10 mins.]
    Students will complete the ``Logic gate review'' worksheet.
    In this worksheet, students write out the truth tables for the four basic
    logic gates: AND, OR, XOR, and NOT.

  \item[15 mins.]
    Students will complete the ``Circuit diagrams and boolean logic''
    worksheet.
    Each group will receive two copies of the worksheet and the worksheet is to
    be completed in pairs.

  \item[30 mins.]
    Students will complete the ``Breadboard reverse engineering worksheet''.
    This is a collaborative activity in which each group is assigned a logical
    circuit to implement physically using transistors and integrated circuits.

  \item[10 mins.]
    The rest of the class will introduce Haskell and programming languages. To
    tie in pre-existing knowledge of hardware, students will understand
    compilation to be moving down the latter of abstraction. Participants are
    now prepared to complete the \url{www.tryhaskell.org} by the next session.
\end{itemize}

\section*{Homework}

Students should build an intuitive understanding of Haskell syntax and types
by completing \url{www.tryhaskell.org} up to but not including lesson 4 (on
pattern matching).

\end{document}
