\documentclass[11pt]{article}
\usepackage[margin=2cm]{geometry}
\usepackage{hyperref}
\title{1 - Binary and logic}
\author{Eric Mayhew \& Jacob Errington}
\date{}
\newcommand{\cwurl}{https://www.computing-workshop.com/}
\newcommand{\cwpdf}{\cwurl pdf/}

\begin{document}
\maketitle
This lesson provides its participants with an introduction to circuits and
binary. Understanding binary and circuits are at the soul of computers,
therefore it is essential they are discussed in order to be comfortable with
computers. By implementing logic gates using breadboards, participants will learn how basic circuits are
constructed using hands-on learning activities. Understanding circuitry will lay the ground work for
introducing more complicated concepts like integrated circuits and processors in later lessons.

\section*{Overview}

\begin{description}
  \item[Lesson Objectives] ~

   By the end of the lesson, students will be able to:

  \begin{itemize}
    \item Explain the function of the power supply unit in the computer.

    \item Build basic logic gates out of transistors, both theoretically and
      physically on a breadboard.

    \item Name a few basic Haskell types and use an interactive Haskell prompt
      like GHCi.
  \end{itemize}

  \item [Materials]~

    To run this lesson, the following are needed:

    \begin{itemize}
      \item
        lesson slides;

        \url{\cwpdf 1-presentation.pdf}

      \item
        one breadboard for each group and enough of the following electronic
        components:
        NPN transistors,
        10~$\Omega$ resistors,
        9~V batteries,
        5~V voltage regulators,
        jumper wires (male to male),
        and assorted LEDs.

    \end{itemize}
\end{description}

\section*{Instructional Sequence}

\begin{itemize}
  \item[5 mins.]
    Facilitators welcome the group and present lesson agenda.

  \item[25 mins.]
    Using the slides, facilitators will present the power supply unit, which
    turns alternating current to direct current, a form of electricity computers
    can use. To represent this, students will implement a basic circuit on their
    breadboard.

    Before creating the circuit, facilitators will explain the purpose of each
    component using the slides.

    The students will now create the circuit shown on slide 10.
    Building this circuit will require: the breadboard, a resistor, voltage
    regulator (VR), the battery, and a LED. The circuit will run a current
    through the VR, then the LED, then a resistor and back to the batter.
    The LED light will to turn it on, demonstrating a basic circuit.
    Students should add a button anywhere before the LED on their circuit to
    get familiar with its function.

  \item[20 mins.]
    Once students feel comfortable with using the breadboard to create
    circuits, facilitators will introduce transistors. For our purpose,
    transistors act as electronic switches which are used to manipulate binary
    values.

    Students will add a transistor to their breadboard. They will do this by
    moving the output of the LED current into the base (middle prong) of the
    transistor. Then they will connect the collector (left prong) to a
    different colour LED and to the positive voltage. Finally the emitter
    (right prong) will connect to the negative voltage.

    Pressing the button of the base wire will light up both the base LED and
    collector LED. Explain this is because the base makes the transistor
    conductive, allowing electricity to move from collector to emitter. The
    base is like a spigot on a tap: when it is open, water flows; when it is
    closed, water does not flow.

  \item[35 mins.]
    When students are comfortable with transistors, they will use 2 transistors
    to make the "and" \& "or" logic gate. To create an and gate using
    transistors, refer to this video:
    \url{https://www.youtube.com/watch?v=sTu3LwpF6XI&t=488s}.

  \item[10 mins.]
    The rest of the class will introduce Haskell and programming languages. To
    tie in pre-existing knowledge of hardware, students will understand
    compilation to be moving down the latter of abstraction. Participants are
    now prepared to complete the \url{www.tryhaskell.org} by the next session.
\end{itemize}

\section*{Homework}

Students should build an intuitive understanding of Haskell syntax and types
by completing \url{www.tryhaskell.org} up to but not including lesson 4 (on
pattern matching).

\end{document}
