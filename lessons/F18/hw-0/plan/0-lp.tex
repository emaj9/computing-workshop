\documentclass[11pt]{article}
\usepackage[margin=2cm]{geometry}
\usepackage{hyperref}
\title{0 - First look at hardware}
\author{Eric Mayhew \& Jacob Errington}
\date{Fall 2018}
\newcommand{\cwurl}{https://www.computing-workshop.com/lesson/pdfs/}

\begin{document}
\renewcommand{\abstractname}{\vspace{-\baselineskip}}
%%%This will remove the "Abstract" title from abstract
\maketitle
\begin{abstract}
  This lesson introduces the workshop to the participants. Starting with
  disassembling a computer, participants will get a bird's eye view as to the
  content that will be covered in the course, mainly the essential components of
  an ordinary computer. We will start our journey to understanding computers
  from the bottom up, looking at electricity and circuits on breadboards.
\end{abstract}
\section*{Overview}
\begin{description}
  \item [Lesson Objectives]
    ~

   By the end of the lesson, students will be able to:
  \begin{itemize}

    \item Identify and describe the function of common computer components,

    \item Disassemble a computer,

    \item Describe how a computer works through abstraction,

    \item Use discrete electrical components to make a simple circuit on a breadboard.

  \end{itemize}
  \item [Materials]~

Here's what you'll need to run this lesson:
  \begin{itemize}
    \item
      lesson slides;

      \url{\cwurl 0-slides.pdf}

    \item
      name tags for participants;

    \item
      computers for disassembly and tools for disassembly, ideally 1
      computer for 4 students;

    \item
      one breadboard for each group and enough of the following electronic
      components:
      NPN transistors,
      10~$\Omega$ resistors,
      9~V batteries,
      5~V voltage regulators,
      jumper wires (male to male),
      and assorted LEDs.

  \end{itemize}
\end{description}
  \section*{Instructional Sequence}
  \begin{itemize}
    \item[10 minutes] Introduce course website, explain where to find lesson
      plans/syllabus and resources, introduce facilitators.
    \item[25 mins.] Assign students to groups of four, where participants will
      play ``two truths and a lie'' and make name tags for themselves. They will
      also decide on a group name.
    \item[20 mins.] Present groups with the computer they will disassembly,
      the tools to disassemble, and the list of components to find (on slide number).
      Circulate amongst the groups to ensure they're on the right track. If
      they need help identifying parts, use images online to help them.
    \item[10 mins.] Get each group to show one component that they found in the
      computer, along with its functionality. This should cover all the
      components found in the computers.
    \item[10 mins.] Next, the class will explore just how these components work
      together. Using the slides, facilitators will introduce the concept of
      ``abstraction'' and its relation to computer science. Abstraction can be
      understood as skimming the surface of all the layers and complexities of
      how a computer works without diving into the nitty gritty details. Using
      the slides, introduce the concept using the ladder of abstraction in the
      slides, highlighting that this workshop will
      focus on the physical side of how computers work.
    \item[10 mins.] This workshop will focus on each core component of a
      computer, starting with the power supply unity (PSU) as our
      introduction. The PSU converts alternating current electricity to direct
      current (DC). DC is very useful for representing the binary state of 1 and
      0.
    \item[20 mins.] In order to get familiar with electricity, participants will
      construct a basic circuit using their breadboards and various electronic
      components. Facilitators will introduce the breadboard and discrete
      electronic components participants will use. Following the slides provides
      said descriptions along with a step-by-step tutorial as to how to make a
      simple circuit to power an LED on a breadboard.
    \item[5 mins.] Once the breadboards are complete, the lesson is
      concluded. Review the concepts covered in class and provide the
      participants with what is to be expected for next class.
  \end{itemize}
\end{document}
