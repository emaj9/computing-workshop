\documentclass[11pt]{article}

\usepackage[margin=2.0cm]{geometry}
\usepackage{amsmath}
\usepackage{tikz}
\usepackage{pgfplots}
\usepackage{pgfplotstable}
\usepackage{booktabs}

\title{$k$ nearest neighbours}
\author{Computing Workshop: Software}

\begin{document}

\maketitle

\begin{figure}[h]
  \centering
  \begin{tikzpicture}
    \begin{axis}[
        grid=major,
        minor x tick num=1,
        minor y tick num=4,
        xlabel={tumor radius},
        ylabel={tumor texture},
        scatter/classes={
          M={mark=square*, red},
          B={mark=triangle*, blue}
        }
      ]
      \addplot[scatter, only marks, scatter src=explicit symbolic]
      table[x=radius, y=texture, meta=diagnosis, col sep=comma] {small-nice.data};
    \end{axis}
  \end{tikzpicture}
  \caption{%
    This plot shows 20 data points from the UCI Breast Cancer Dataset.
    The dataset contains measurements from people with breast tumors, some
    benign and some malignant. Malignant tumors are dangerous, so it's important
    to be able to predict whether a tumor might be malignant.
    The full dataset contains 10 different measurements per person, so the data
    is $10$-dimensional! To make things printable on a sheet of paper, we focus
    on only two measurements: the tumor radius ($x$ axis) and the tumor texture
    ($y$ axis).%
    Malignant tumors are shown as squares (red) and benign tumors are shown as
    triangles (blue).
  }
  \label{fig:chart}
\end{figure}

\section*{Questions}

Suppose we want to classify a new patient, whose measurements are
$\text{tumor radius} = 15$ and $\text{tumor texture} = 18$.

\begin{enumerate}
\item
  By visual inspection of figure \ref{fig:chart}, what classification would this
  data point be given?
\item
  By looking at the 
\end{enumerate}

\begin{figure}[h]
  \centering
  % \pgfplotstableread[col sep=comma]{small-nice.data}\mydata
  \pgfplotstableread[col sep=comma]{small-nice.data}\smallnice
  \pgfplotstableset{
    create on use/Distance/.style={
      create col/expr={%
        sqrt((\thisrow{radius}-15)^2
        +
        (\thisrow{texture}-18)^2)}}}
  \pgfplotstabletypeset[
    columns={id, diagnosis, radius, texture, Distance},
    every head row/.style={before row=\toprule, after row=\midrule},
    every last row/.style={after row=\bottomrule},
    display columns/0/.style={string type, column name={ID}},
    display columns/1/.style={string type, column name={Diagnosis}},
    display columns/2/.style={column name={Radius}},
    display columns/3/.style={column name={Texture}},
  ]\smallnice
  \caption{%
    The table of data used to generate the preceding scatter plot.
    $M$ indicates a malignant diagnosis and $B$, benign.%
  }
\end{figure}

\end{document}
