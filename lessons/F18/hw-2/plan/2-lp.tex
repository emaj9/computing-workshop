\documentclass[11pt]{article}

\usepackage[margin=2cm]{geometry}
\usepackage{hyperref}

\title{2 -- Integrated circuits and the ALU}
\author{Eric Mayhew \& Jacob Errington}
\date{}
\newcommand{\cwurl}{https://www.computing-workshop.com/}
\newcommand{\cwpdf}{\cwurl pdf/}

\newcommand{\codeworld}{\href{http://code.world/}{code world}}

\begin{document}

\maketitle

This lesson guides its participants up the ladder of abstraction from
transistors to integrated circuits, which bundle up useful logical operations
into their own electronic component. Using the logic operations learned last
lesson, participants will create different features of the arthmetic and logic unit (ALU)
on their breadboards. This lays the foundation for exploring memory, and
eventually the processor in futher lessons.

\section*{Overview}

\subsection*{Learning objectives}

By the end of this lesson, students will be able to:
%
\begin{itemize}
\item Translate numbers from base 10 to base 2;
  \item Describe the function and use of intergrated circuits (ICs) in computing;
\item Describe the common operations of the ALU;
  \item Reconstruct a full adder and other ALU features using ICs on a breadboard.
\end{itemize}

\subsection*{Materials}

To run this lesson, the following materials are necessary:

\begin{itemize}
\item The lesson slides, \cwpdf{2-presentation.pdf}.
\item One breadboard for each group and enough of the following electronic
  components:
  NPN transistors,
  100~$\Omega$ resistors,
  9~V batteries,
  5~V voltage regulators,
  jumper wires (male to male),
  assorted LEDs,
  integrated circuits from the 7400-series (7408 (quad 2-input AND gate), 7432
  (quad 2-input OR gate)).
\end{itemize}

\section*{Instructional sequence}

\begin{itemize}
\item[5 mins.]
  Facilitators will greet the participants and present the lesson agenda.
\item[10 mins.]
  To get warmed up, each group will reconstruct a truth table from the logic gate of their choosing
  covered last week. Groups are also encouraged to research and construct the
  truth table for logic gates we didn't get to cover, like the NAND and NOR
  gates. Each group will present their truth table, explaining how it
  works. It's important these truth tables stay on the board as they will be
  used later in the lesson.
\item[10 mins.]
  Facilitators will introduce integrated circuits as packaged-up
  transistor-based circuits, including the history of the integrated
  circuit. Participants must understand that ICs are an abstraction of
  transistor-based logic gates. In this lesson, the class will reconstruct parts
  of an ALU on their breadboards.
\item[10 mins.]
  Starting with the arithmetic part of the ALU, facilitators will review the
  relationship between binary and base 10. Using the activites outlined in the
  slides, participants will translate base 2 numbers into base 10.
\item[10 mins.]
  Next, the participants will explore base 2 addition. To start, one volunteer
  with the the help of the facilitator will review how to add two numbers in
  base 10. An example might be something like 99 + 2. In this example, the
  facilitator should emphasize what happens to the ``carry over'' of 99 + 2,
  namely that a new column is added to the right of the sum (as in the 1 in 100
  is a new ``column'' in the number). Next, the participant will add two binary
  numbers such as 01 and 11. Working from the right most column, 1 + 1 obviously
  equals 2, but in binary there is no 2, so the sum is actually 10. It's
  important to note that this isn't ten like in base 10, but rather the sum of 1
  + 1 is 0 carry 1. This ``carry 1'' carries the value 1 to the next column to
  the left. There for the next column is adding 0 + 1 + the carry 1 from the
  right column. Here, the same principle applies where 1 + 1 in binary results
  in 0 carry 1. Thus, in binary, 01 + 11 = 110.
\item[10 mims.] Now with an intuition as to how binary addition works,
  participants will consider how to structure their breadboard to add any
  combination of numbers in binary, in other words, create a half-adder. To do
  this, a volunteer will write all the possible combinations of adding two bits
  together in a table. Then, participants will assess which truth table written
  on the board earlier in the lesson correlates closest to the possible
  combinations of the single bit addition. The closest is the XOR gate, with the
  exception being that when both inputs are 1, the output should not be 1 but
  instead 10. Another logic gate will have to be introduced to our system to
  account for this outcome, and luckily the AND gate does just that! Therefore
  the AND and XOR gate together create our bit adder.
\item[10 mins.] Before diving into creating the half adder on their breadboards,
  participants will learn about the history and use of ICs using the
  slides. These slides provide important information as to how to use ICs.
\item[10 mins.] Once ICs are clear, participants will implement the half adder
  to their breadboard.
\item[20 mins.] A half adder isn't enough if we want to add more than two,
  single bit inputs. In order to add larger numbers, a full adder must be
  implemented. Referencing the slides and video, participants will get an
  understanding how to implement the full adder and create one on their
  breadboards.
\item[20 mins.] Addition is not the only feature of a breadboard
  however. Facilitators will introduce an example feature, such as checking if
  an input is all zero using OR gates. Afterwards, participants will be given a
  list of common ALU features to implement using breadboards to conclude the
  lesson.
\item[5 mins.] To conclude, facilitators will review the concepts covered in
  this class as well as what is to be expected next week.
\end{itemize}

\end{document}
