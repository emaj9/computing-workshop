\documentclass[11pt]{article}

\usepackage[margin=2.0cm]{geometry}
\usepackage{hyperref}

\newcommand{\code}{\texttt}

\author{%
  Eric Mayhew \& Jacob Errington%
  \footnote{%
    Special thanks to Building21 (McGill's Office of Student Life and Learning)
    and to Anita Parmar.
    Computing Workshop would not have been possible without her tremendous
    support!
  }
}
\title{Computing Workshop: Hardware Syllabus}
\date{Fall 2018}

\begin{document}

\maketitle

\begin{description}
  \item[Website:]
    \url{http://computing-workshop.com/}

  \item[Location:]
    B21, 651 rue Sherbrooke Ouest
    (Northeast corner of University street)

  \item[Time:]
    Monday from 2:00PM to 4:00PM, on
    \begin{itemize}
    \item September 10, 17, 24,
    \item October 1, (no class on Thanksgiving), 15, and 22.
    \end{itemize}

  \item[Goal:]
    At the end of the workshop, particpants will be able to identify the core
    hardware components of a modern computer and describe their function.
    They will be able to construct simple hardware components using discrete
    circuitry and describe how these components work together to form
    a computer.
\end{description}

\section*{Description}

Computing Workshop: Hardware Unit is the half of Computing Workshop that focuses
on the physical components of computers. This unit introduces to participants to binary and elementary circuitry elements
such as transistors, eventually leading up to recreating memory and arithmetic and logic unit of a CPU. The workshop
climbs the ladder of abstraction, providing participants with a high level understanding of processors and operating systems.

\section*{Rationale}

We created this workshop to provide people with the support to feel comfortable
using a computer. Technology is pervasive in our society, but rarely are
individuals provided with meaningful education about how to they work. Our workshop focuses on hands on learning.
Circuits are implement on breadboards and larger processes are simulate with group activities, allowing students to
create their own understanding of the hardware of a computer.

\vfill
\section*{Lesson sequence}

\begin{enumerate}
    \setcounter{enumi}{-1}
  \item Intro to circuits and electricity
  \item Logic and the transistor
  \item Intergrated circuits (ICs), the half adder, and ALU
  \item Memory
  \item Processor
  \item Operating systems and conclusion
\end{enumerate}

\section*{Learning goals}

\subsection*{Computers}

\begin{itemize}
  \item identify key hardware components of a computer;
  \item effectively use the electronic components necessary in creating circuits;
  \item describe each hardware component's function;
  \item recreate basic hardware components using circuits;
  \item identify core services provided by an operating system;
\end{itemize}

\end{document}
