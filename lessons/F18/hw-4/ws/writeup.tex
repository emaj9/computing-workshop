\documentclass[11pt]{article}

\usepackage[margin=1.6cm]{geometry}

\title{Computing Workshop: Hardware\\Processor}

\newcommand{\regname}{\textbf}
\newcommand{\instname}{\textit}

\begin{document}

\maketitle

\section*{Instructions}

Recall the roles:
\begin{description}
\item[Program Counter.]
  Manages all the registers, including the \emph{instruction address register}
  aka the \emph{program counter}.
  Initially, all registers (including the program counter) are zero.
\item[Fetcher.]
  The Program Counter tells the Fetcher to read data from memory.
\item[Decoder.]
  The Decoder keeps their hands on the below Instruction Table in order to
  decode the instruction fetched from memory.
  The Decoder communicates with Program Counter and Fetcher to retrieve
  additional data needed to perform the instruction and to write data into
  registers if needed.
  The decoder also commands the ALU to perform arithmetic and logical operations.
\item[ALU.]
  The ALU performs the operations demanded of it.
\end{description}

How to perform the fetch-decode-execute cycle.
\begin{enumerate}
\item
  The Program Counter asks the Fetcher to retrieve two bytes from memory
  starting at the address in the instruction address register.
\item
  The Fetcher gives these bytes to the Decoder, who decodes the instruction.
\item
  Depending on the instruction, the Decoder will ask the Program Counter to
  write values into registers, or ask the Fetcher to read one byte from memory,
  or ask the ALU to perform an arithmetic operation.
\item
  Repeat until you reach the Halt instruction!
\end{enumerate}

\newpage

\section*{Instruction table}

\begin{figure}[h]
  \centering
  \begin{tabular}{c c c c}
    \textbf{Operation} & \textbf{Op code} &
    \textbf{Register Operand (4 bits)} & \textbf{Address Operand (8 bits)} \\
    %
    Direct Load &
    $0000$ & Destination register & Address to load (from RAM) \\
    %
    Direct Store &
    $0001$ & Source register & Destination address (in RAM) \\
    %
    Indirect Load &
    $0010$ & Destination register & Source register (last 4 bits) \\
    %
    Indirect Store &
    $0011$ & Source register & Destination register (last 4 bits) \\
    %
    Add to Register  &
    $0100$ & Destination register & Source registers \\
    %
    XOR to Register &
    $0101$ & Destination register & Source registers \\
    %
    Register Copy to Register &
    $0110$ & Destination register & Source register (last 4 bits) \\
    %
    Value Copy to Register &
    $0111$ & Destination register & Value \\
    %
    Unconditinal Jump &
    $1000$ & -- & Address to jump to \\
    %
    Jump if Zero &
    $1001$ & -- & Address to jump to \\
    %
    Compare Value to Register &
    $1101$ & Register to compare & Value to compare to \\
    %
    Halt (end) &
    $1111$ & -- & --
  \end{tabular}
  \caption{%
    Reference instruction table. Unused operands are marked by a dash ``--''.
    %
  }
\end{figure}

\section*{Instruction descriptions}

\begin{description}
\item[Direct Load \& Direct Store.]
  A \emph{Direct Load} or \emph{Direct Store} instruction operates on an exact
  memory address specified in the Address Operand.
  For example, 0001 0000 0010 0110 stores the value of register 0000 into memory
  address 00100110.

\item[Indirect Load \& Indirect Store.]
  An \emph{Indirect Load} or \emph{Indirect Store} instruction reads from or
  writes to memory, but uses an address contained in a register rather than a
  predetermined address.
  This way, programs can compute addresses and operate on addresses that are
  unknown at the time the program is written.
  In order to specify the register containing the address, the lower (rightmost)
  4 bits.

  For example, 0010 0100 0000 0101 indirectly loads the value at the address
  contained in register 0101 into register 0100.

\item[Arithmetic \& logical operations.]
  The Add and XOR operations operate on registers. They perform the operation on
  the two registers specified in the Address Operand and the result is stored in
  the register specified in the Register Operand.

\item[Copy to Register.]
  The Register Copy to Register instruction copies the value from a source
  register, specified in the low half (rightmost 4 bits) of the Address Operand,
  to a destination register, specified in the Register Operand.

  The Value Copy to Register instruction copies a literal value contained in the
  Address Operand into the register specified in the Register Operand.

\item[Jumps.]
  The Unconditional Jump instruction sets the program counter to the value
  contained in the Address Operand.

  The Jump if Zero instruction sets the the program counter to the value
  contained in the Address Operand if the ALU's Zero Flag is set, i.e. when the
  result of a preceding arithmetic or logical operation is zero.

  Note that if a conditional jump does not execute (because its condition is not
  satified), then the program counter simply increments as usual.

\item[Compare.]
  The compare instructions check whether two things are equal, e.g.  two
  registers are equal or whether a register equals a given value.
  If the comparison is successful (the things are in fact equal), then the ALU's
  Zero Flag is set.
\end{description}

\end{document}
