\documentclass[11pt]{article}

\usepackage[margin=2.0cm]{geometry}

\title{Computing Workshop: Hardware -- Memory}

\newcommand{\num}[1]{$\mathbf{#1}$}

\begin{document}

\maketitle

\begin{figure}[h]
  \centering
  \begin{tabular}{l||c|c|c|c|c|c|}
    ~ & \num{2} & \num{3} & \num{4} & \num{5} & \num{6} & \num{7} \\ \hline\hline
    \num{0} & \verb| | & \verb|0| & \verb|@| & \verb|P| & \verb|`| & \verb|p|  \\ \hline 
    \num{1} & \verb|!| & \verb|1| & \verb|A| & \verb|Q| & \verb|a| & \verb|q|  \\ \hline 
    \num{2} & \verb|"| & \verb|2| & \verb|B| & \verb|R| & \verb|b| & \verb|r|  \\ \hline 
    \num{3} & \verb|#| & \verb|3| & \verb|C| & \verb|S| & \verb|c| & \verb|s|  \\ \hline 
    \num{4} & \verb|$| & \verb|4| & \verb|D| & \verb|T| & \verb|d| & \verb|t|  \\ \hline 
    \num{5} & \verb|%| & \verb|5| & \verb|E| & \verb|U| & \verb|e| & \verb|u|  \\ \hline 
    \num{6} & \verb|&| & \verb|6| & \verb|F| & \verb|V| & \verb|f| & \verb|v|  \\ \hline 
    \num{7} & \verb|'| & \verb|7| & \verb|G| & \verb|W| & \verb|g| & \verb|w|  \\ \hline 
    \num{8} & \verb|(| & \verb|8| & \verb|H| & \verb|X| & \verb|h| & \verb|x|  \\ \hline 
    \num{9} & \verb|)| & \verb|9| & \verb|I| & \verb|Y| & \verb|i| & \verb|y|  \\ \hline 
    \num{A} & \verb|*| & \verb|:| & \verb|J| & \verb|Z| & \verb|j| & \verb|z|  \\ \hline
    \num{B} & \verb|+| & \verb|;| & \verb|K| & \verb|[| & \verb|k| & \verb|{|  \\ \hline
    \num{C} & \verb|,| & \verb|<| & \verb|L| & \verb|\| & \verb|l| & \verb|||  \\ \hline
    \num{D} & \verb|-| & \verb|=| & \verb|M| & \verb|]| & \verb|m| & \verb|}|  \\ \hline
    \num{E} & \verb|.| & \verb|>| & \verb|N| & \verb|^| & \verb|n| & \verb|~|  \\ \hline
    \num{F} & \verb|/| & \verb|?| & \verb|O| & \verb|_| & \verb|o| & \verb|DEL|\\ \hline
  \end{tabular}
  \caption{%
    The ASCII table. Use this table to decode a hexadecimal number into a
    character. To distinguish hexadecimal numbers from decimal numbers, we write
    hex numbers with the prefix \texttt{0x}. For example, to decode
    \texttt{0x61}, look in the column labelled \texttt{6} and the row labelled
    \texttt{1}.
  }
\end{figure}

% The RAM is supposed to store:
% 
\begin{figure}[h]
  \centering
  \begin{tabular}{l||c|c|c|c|c|c|c|c|}
    ~ & \num{0} & \num{1} & \num{2} & \num{3} & \num{4} & \num{5} & \num{6} & \num{7} \\ \hline \hline
    \num{ 0 } & $00100101$ & $01101000$ & $01100101$ & $01101100$ & $01101100$ & $01101111$ & $00110111$ & $00111000$ \\ \hline
\num{ 1 } & $01111000$ & $00111001$ & $01110001$ & $00110001$ & $00110111$ & $01011111$ & $00111010$ & $00001010$ \\ \hline
\num{ 2 } & $00100110$ & $00111011$ & $00001010$ & $00100001$ & $00000000$ & $00001010$ & $00111100$ & $00110011$ \\ \hline
\num{ 3 } & $01010100$ & $01011111$ & $01010100$ & $00111011$ & $01100011$ & $01101111$ & $01101101$ & $01110000$ \\ \hline
\num{ 4 } & $01110101$ & $01110100$ & $01101001$ & $01101110$ & $01100111$ & $00101101$ & $01110111$ & $01101111$ \\ \hline
\num{ 5 } & $01110010$ & $01101011$ & $01110011$ & $01101000$ & $01101111$ & $01110000$ & $00100001$ & $00100001$ \\ \hline
\num{ 6 } & $01101101$ & $01101111$ & $01101010$ & $01101001$ & $01100010$ & $01100001$ & $01101011$ & $01100101$ \\ \hline
\num{ 7 } & $00010011$ & $00001000$ & $00001111$ & $00010000$ & $00010111$ & $00101010$ & $00111111$ & $00111111$ \\ \hline

  \end{tabular}
  \caption{%
    The initial contents of our memory (aka RAM), represented in binary.
    Since our RAM has eight rows and eight columns, we need \emph{six} bits to
    store an address: three for the row number and three for the column number.
    For example, consider the address \texttt{110010}. This refers to the cell
    in column $010_2 = 2$ and row $110_2 = 6$, which is storing the number
    $01101010_2$.%
  }
\end{figure}

\section*{Exercises}

\begin{enumerate}
\item
  Compute the sum of the values in the cells referred to by the addresses
  $111000$ and $111100$.
  Overwrite the value in cell $001101$ with the value you computed.
  \vspace{4em}

\item
  Using the ASCII table, decode eight bytes starting at address $110000$. This
  spells a japanese word; look it up!
  \vspace{4em}

\item
  Use the ASCII table to decode the sequence of bytes starting at address
  $001101$ and continuing until you read a \texttt{NUL} character.

  Don't worry if what you decode is just symbols! Hint: if you read the symbols
  aloud, it should form a kind of poem.
  \vspace{4em}
\end{enumerate}

\end{document}
