\documentclass[11pt]{article}

\usepackage[margin=2.0cm]{geometry}
\usepackage{hyperref}

\newcommand{\code}{\texttt}

\author{%
  Eric Mayhew \& Jacob Errington%
  \footnote{%
    Special thanks to Building21 (McGill's Office of Student Life and Learning)
    and to Anita Parmar.
    Computing Workshop would not have been possible without her tremendous
    support!
  }
}
\title{Computing Workshop: Software Syllabus}
\date{Fall 2018}

\begin{document}

\maketitle

\begin{description}
  \item[Website:]
    \url{http://computing-workshop.com/}

  \item[Location:]
    B21, 651 rue Sherbrooke Ouest
    (Northeast corner of University street)

  \item[Time:]
    Monday from 2:00PM to 4:00pm on
    \begin{itemize}
    \item October 29,
    \item November 5, 12, 19, 26, and
    \item December 3.
    \end{itemize}

  \item[Goal:]
    At the end of the workshop, participants will also be able to modify and implement: decision tree, k-nearest
    neighbour, and neural network classification algorithms using the Python library scikit learn.

\end{description}

\section*{Description}

Computing Workshop: Software Unit is one of two units in Computing Workshop that
focuses on the software side of computers. This unit focuses on machine
learning, a field of computer science that allows computer to ``learn''. This
means improving a computers performance on a given task without hardcoding the
software. Using Python libraries to understand and implement these algorithms,
participants will interact with pre-existing machine learning libraries, modify
them, and eventually create their own algroithms!

\section*{Rationale}

We created this workshop to provide anyone interested in machine learning with a
guided first step into the subject. Machine learning is a highly popular field
of computer science, and for good reason. It has very pracitcal uses in a wide
range of fields, from financial market analysis, to algriculture, to art history to name a few of the many of
disciplines that machine learning
applies to. This workshop aims to provide participants with little to no coding
background with a solid introduction to machine learning using hands on
acitivites, laying the foundation for further exploration of the topic.

\vfill
\section*{Lesson sequence}

\begin{enumerate}
    \setcounter{enumi}{-1}
  \item Intro to machine learning
  \item K-nearest neighbour and algorithms
  \item Python and decision trees
  \item Neural networks
  \item Machine learning lab
  \item Other kinds of machine learning and conclusion
\end{enumerate}

\section*{Learning goals}

By the end of the workshop, participants will be able to:

\begin{itemize}
  \item justify the use of a particular machine learning algorithm to solve a problem;
  \item implement the machine learning algorithms covered in the workshop using scikit learn;
  \item describe at a high level of a function of the machine learning algorithms covered in the workshop;
  \item find and interact with machine learning repositories and library documentation;
  \item use python functions, data structures, and control flow to
    implement machine learning alrogrihtms.
\end{itemize}

\end{document}
