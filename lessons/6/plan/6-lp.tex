\documentclass[11pt]{article}

\usepackage[margin=2cm]{geometry}
\usepackage{hyperref}

\title{6 -- HTML}
\author{Eric Mayhew \& Jacob Errington}
\date{}
\newcommand{\cwurl}{https://www.computing-workshop.com/}
\newcommand{\cwpdf}{\cwurl pdf/}

\newcommand{\codeworld}{\href{http://code.world/}{code world}}

\begin{document}

\maketitle

With enough general knowledge of computers, participants are ready to undergo
the project of creating their own website! In order to start creating a website,
participants will plan out their site as well as learn how to locally host a
webpage they're working on. By the end of the lesson, participants will have be
able to locally host their landing page for their site written in HTML.

\section*{Overview}

\subsection*{Learning objectives}

By the end of this lesson, students will be able to:
%
\begin{itemize}
\item Physically construct a NOT gate using a discrete transistor on a breadboard.
\end{itemize}

\subsection*{Materials}

To run this lesson, the following materials are necessary:

\begin{itemize}
\item The lesson slides, \cwpdf{2-presentation.pdf}.
\end{itemize}

\section*{Instructional sequence}

\begin{itemize}
\item[5 mins.]
  Facilitators will greet the participants and present the lesson agenda.
\end{itemize}

\section*{Homework}
\end{document}
