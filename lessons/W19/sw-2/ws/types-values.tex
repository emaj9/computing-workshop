\documentclass[11pt]{article}

\author{Computing Workshop}
\title{Worksheet \#1} 
\date{Fall 2018}

\usepackage{listings}
\usepackage[margin=2.0cm]{geometry}

\lstset{
  basicstyle=\ttfamily
}

\begin{document}

\maketitle

\section*{Types and values}

For the following expressions:
\begin{enumerate}
  \item
    Guess the type and result of evaluating the snippet.
  \item
    Test the snippet in your Jupyter notebook.
  \item
    Did you get the right answer? If not, why?
\end{enumerate}

\begin{center}
\renewcommand{\arraystretch}{3.5}
% The thing in braces after tabular are the column specifications.
% The first column is a fixed width column (14 em wide)
% The second column will be big enough to wrap its contents and will
% center-align text.
% The pipe symbols specify that there should be lines between the columns.
\begin{tabular}{| c | p{5em}| p{5em} | p{21em} |}
    \hline % Use hline to make horizontal lines between rows.
    \textbf{Expression} & \textbf{Expected output}
    & \textbf{Expected output type} & \textbf{Were you right or wrong? Why?} \\ \hline
    % Use tilde "~" to create a blank cell.
    % Use & to separate cells.
    ``5'' + 5 & ~ & ~ & ~ \\ \hline
    ``5 + 5'' & ~ & ~ & ~ \\ \hline
    ``five'' + ``five'' & ~ & ~ & ~ \\ \hline
    not True == not True & ~ & ~ & ~ \\ \hline
    not (True == True) & ~ & ~ & ~ \\ \hline
    5 + 5.0 & ~ & ~ & ~ \\ \hline
    2/3 & ~ & ~ & ~ \\ \hline
\end{tabular}
\end{center}

\newpage

\section*{Variables \& data structures}

For each code snippet:
\begin{enumerate}
  \item
    Guess the type and result of evaluating the snippet.
  \item
    Test the snippet in your Jupyter notebook.
  \item
    Did you get the right answer? If not, why?
\end{enumerate}
%
Consider these code snippets:
\begin{enumerate}
  \item
    \lstinputlisting{snippet1.py}

  \item
    \lstinputlisting{snippet2.py}

  \item
    \lstinputlisting{snippet3.py}

  \item
    \lstinputlisting{snippet4.py}
\end{enumerate}

\renewcommand{\arraystretch}{3.5}
\begin{tabular}{| c | p{5em}| p{5em} | p{21em} |}
    \hline % Use hline to make horizontal lines between rows.
    \textbf{Snippet} & \textbf{Expected output}
    & \textbf{Expected output type} & \textbf{Were you right or wrong? Why?} \\ \hline
    % Use tilde "~" to create a blank cell.
    % Use & to separate cells.
    Snippet \#1 & ~ & ~ & ~ \\ \hline
    Snippet \#2 & ~ & ~ & ~ \\ \hline
    Snippet \#3 & ~ & ~ & ~ \\ \hline
    Snippet \#4 & ~ & ~ & ~ \\ \hline
\end{tabular}

\end{document}
