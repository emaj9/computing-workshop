\documentclass[11pt]{article}

\author{Computing Workshop: Machine Learning}
\title{Decision trees}
\date{Winter 2019}

\usepackage[margin=2.0cm]{geometry}

\begin{document}

\maketitle

\section*{Questions}

\begin{enumerate}
\item
  What is the interpretation of the colors in the tree visualization?
  \vspace{2em}

\item
  In each node, what is the relationship between the counts in the
  \texttt{value} list and \texttt{samples}?
  \vspace{2em}

\item
  Notice that in all the teal and purple nodes, the first count in
  \texttt{value} is always zero. Why is that?
  \vspace{2em}

\item
  What is the interpretation of the \texttt{value} entry in each node?
  \vspace{2em}

\item
  Invent three data points.
  Choose feature values so that the first point falls into the orange leaf; the
  second, into the teal leaf; and the third, into the most purple leaf. Recall
  that leaves are nodes with no children.
  For example, a point with ``petal width = $2.0$'' would follow the False
  branch of the root node. Choose the other parameters so that the point arrives
  at the desired classification.

  \begin{center}
    \renewcommand{\arraystretch}{1.5}
    % The thing in braces after tabular are the column specifications.
    % The first column is a fixed width column (14 em wide)
    % The second column will be big enough to wrap its contents and will
    % center-align text.
    % The pipe symbols specify that there should be lines between the columns.
    \begin{tabular}{|p{14em}|c|}
      \hline % Use hline to make horizontal lines between rows.
      \textbf{Data point} & \textbf{Desired classification} \\ \hline
      % Use tilde "~" to create a blank cell.
      % Use & to separate cells.
      ~ & Orange \\ \hline
      ~ & Teal \\ \hline
      ~ & Purple \\ \hline
    \end{tabular}
  \end{center}
\end{enumerate}
\end{document}
