\documentclass[11pt]{article}

\usepackage{float}
\usepackage[margin=2.0cm]{geometry}
\usepackage{amsmath}
\usepackage{tikz}
\usepackage{pgfplots}
\usepackage{pgfplotstable}
\usepackage{booktabs}

\title{$k$ nearest neighbours}
\author{Computing Workshop: Software}

\begin{document}

\maketitle

\begin{figure}[H]
  \centering
  \begin{tikzpicture}
    \begin{axis}[
        grid=major,
        minor x tick num=1,
        minor y tick num=4,
        xlabel={tumor radius},
        ylabel={tumor texture},
        scatter/classes={
          M={mark=square*, red},
          B={mark=triangle*, blue}
        },
        legend cell align=left,
        legend pos=north west,
      ]
      \addplot[scatter, only marks, scatter src=explicit symbolic]
      table[x=radius, y=texture, meta=diagnosis, col sep=comma] {small-nice.data};
      \legend{Malignant,Benign};
    \end{axis}
  \end{tikzpicture}
  \caption{%
    This plot shows 20 data points from the UCI Breast Cancer Dataset.
    The dataset contains measurements from people with breast tumors, some
    benign and some malignant. Malignant tumors are dangerous, so it's important
    to be able to predict whether a tumor might be malignant.
    The full dataset contains 10 different measurements per person, so the data
    is $10$-dimensional! To make things printable on a sheet of paper, we focus
    on only two measurements: the tumor radius ($x$ axis) and the tumor texture
    ($y$ axis).%
    Malignant tumors are shown as squares (red) and benign tumors are shown as
    triangles (blue).
  }
  \label{fig:chart}
\end{figure}

\newpage

\section*{Questions}

Suppose we want to classify a new patient $p$ whose measurements are
$\text{tumor radius} = 15$ and $\text{tumor texture} = 18$.

\begin{enumerate}
\item
  By visual inspection of figure \ref{fig:chart}, what classification would you
  give the point $p$?
  \vspace{2em}
\item
  By using the table in figure \ref{fig:table}, decide on a classification using
  the $k$ nearest neighbours of $p$ for $k=1,2,3,5$. For example, if the majority
  of the nearest neighbors are malignant, then classify $p$ as malignant.
  \vspace{2em}
\end{enumerate}

\begin{figure}[H]
  \centering
  % \pgfplotstableread[col sep=comma]{small-nice.data}\mydata
  \pgfplotstableread[col sep=comma]{small-nice.data}\smallnice
  \pgfplotstableset{
    create on use/Distance/.style={
      create col/expr={%
        sqrt((\thisrow{radius}-15)^2
        +
        (\thisrow{texture}-18)^2)}}}
  \pgfplotstabletypeset[
    columns={id, diagnosis, radius, texture, Distance},
    every head row/.style={before row=\toprule, after row=\midrule},
    every last row/.style={after row=\bottomrule},
    display columns/0/.style={string type, column name={ID}},
    display columns/1/.style={string type, column name={Diagnosis}},
    display columns/2/.style={column name={Radius}},
    display columns/3/.style={column name={Texture}},
  ]\smallnice
  \caption{%
    The table of data used to generate the preceding scatter plot.
    $M$ indicates a malignant diagnosis and $B$, benign.
    Included in this table is also the distance of each row to the new patient
    $p$. %
  }
  \label{fig:table}
\end{figure}

\newpage

\section*{More Questions}

Suppose we use a $3$ nearest neighbors algorithm to classify $20$ points as
either benign or malignant.
The table below lists the points to classify as well as their predicted
classification and actual classification.

\begin{enumerate}
\item
  Fill in the \emph{confusion matrix} table below to contrast the classification
  errors made by this classifier.
\item
  Discuss in your groups the consequences of false positives and false negatives
  in the context of breast cancer screening. Are some errors more dangerous than
  others?
\end{enumerate}

\begin{figure}[h]
  \centering
  \pgfplotstableread[col sep=comma]{confusion-data.csv}\confusion
  \pgfplotstabletypeset[
    columns={Radius, Texture, Actual, Prediction},
    every head row/.style={before row=\toprule, after row=\midrule},
    every last row/.style={after row=\bottomrule},
    display columns/2/.style={string type},
    display columns/3/.style={string type},
  ]\confusion
  \caption{%
    The data generated by the $3$ nearest neighbours classifier. Sometimes, it
    doesn't give the right answer!%
  }
  \label{fig:confusion}
\end{figure}

\begin{figure}[h]
  \centering
  \begin{tabular}{l || c | c |}
    ~ & \textbf{Predicted $B$} & \textbf{Predicted $M$} \\ \hline \hline
    \textbf{Actually $B$} & ~~~~~ & ~~~~~ \\ \hline
    \textbf{Actually $M$} & ~~~~~ & ~~~~~ \\ \hline
  \end{tabular}
  \caption{%
    This is the \emph{confusion matrix} of the classifier. It describes what
    types of correct and incorrect classifications were made.%
    In the cell \textbf{Actually $B$ - Predicted $M$}, you should fill in the
    number of times the classifier reported an $M$ when it should have reported
    a $B$. That's the number \emph{false positives} the classifier made.
    Fill in the other cells similarly.
  }
\end{figure}

\end{document}
