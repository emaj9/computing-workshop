\documentclass[11pt]{article}

\usepackage[margin=2.0cm]{geometry}
\usepackage{hyperref}

\newcommand{\code}{\texttt}

\author{%
  Eric Mayhew \& Jacob Errington%
  \footnote{%
    Special thanks to Building21 (McGill's Office of Student Life and Learning)
    and to Anita Parmar.
    Computing Workshop would not have been possible without her tremendous
    support!
  }
}
\title{Computing Workshop: Software Syllabus}
\date{Wall 2019}

\begin{document}

\maketitle

\begin{description}
  \item[Website]
    \url{http://computing-workshop.com/}

  \item[Location]
    B21, 651 rue Sherbrooke Ouest
    (Northeast corner of University street)

  \item[Time]
    Monday from 2:00PM to 4:00pm on
    \begin{itemize}
    \item January 14, 21, 28
    \item February 4, 11, and 18
    \end{itemize}

  \item[Learning Goals]
    At the end of the workshop, participants will be able to:
    \begin{itemize}
      \item Describe at a high level what machine learning is and how it works, the uses and applications of machine learing,
        as well as the limits and ethical implications of machine learning;
      \item Explain in plain English the following algorithms work: k-nearest neighbour, decision trees, neural networks;
        k-means, and DBSCAN;
      \item Implement the following algorithms using the appropriate Python libraries: k-nearest neighbour, decision trees, and neural networks.
    \end{itemize}

\end{description}

\section*{Description}

This instance in Computing Workshop focuses on machine
learning, a field of computer science that allows algorithms to ``learn''. This
means improving a computer's performance on a given task without hardcoding the
improvement. This workshop focuses on a handful of machine learning algorithms, covering how they work in plain English and
how to code them in Python. We hope to give participants a general introduction to machine learning. This workshop ends with
discussing at a high level the limits and ethics of machine learning.

\section*{Rationale}

We created this workshop to provide a guided first step into machine learning to
anyone interested in the subject.
Machine learning is a highly popular field of computer science, and for good
reason:
when algorithms can learn, they can become a very powerful and practical tool
that are increasingly employed in our daily lives.
As these algorithms become more and more pervasive and influential in our
society, it's important that people learn about machine learning so they can be
informed as to the limitations, uses and potential misuses of these alrogithms.
This workshop aims to provide participants with little to no coding
background with a solid introduction to machine learning using hands on
acitivites, laying the foundation for further exploration of the topic.

\section*{Lesson sequence}

\begin{enumerate}
    \setcounter{enumi}{-1}
  \item Intro to machine learning and Python
  \item K-nearest neighbour
  \item Decision trees
  \item Neural networks
  \item Guest speaker and lab
  \item Other kinds of machine learning and ethics of machine learning
\end{enumerate}

\end{document}
