\documentclass[11pt]{article}

\usepackage[margin=2cm]{geometry}
\usepackage{hyperref}

\title{4 - Operating systems and Linux!}
\author{Jacob Errington \& Eric Mayhew}
\date{}

\newcommand{\cwurl}{https://www.computing-workshop.com/}
\newcommand{\cwpdf}{\cwurl pdf/}

\begin{document}

\maketitle

In this lesson, participants will be introduced to operating system
fundamentals. We will explore the boot sequence of a laptop computer in detail,
and present the high-level features of a modern operating system.
Finally, participants will begin to use the Linux command line.

\section*{Overview}

\subsection*{Learning objectives}

By the end of the lesson, students will be able to:
\begin{itemize}
\item
  Explain the boot sequence of a typical modern computer.
\item
  Name some key features of a modern operating system, such as user isolation,
  memory protection, and hardware abstraction.
\item
  Use a (bash-compatible) Linux shell for basic file management such as changing
  directories, moving/copying files, creating files and directories, and viewing
  text files.
\item
  Use an editor from the \texttt{vi} family to edit files
\end{itemize}

\subsection*{Materials}

To run this lesson, the following materials are necessary:

\begin{itemize}
\item
  The slides, available here:
  \url{\cwpdf{4+-+Operating+systems+and+Linux.pdf}}
\item
  Each participant requires a computer with VirtualBox installed, and each
  participant needs to download and unzip the prepared virtual machine:
  \url{https://files.jerrington.me/arch.zip}.

\item
  In particular, this prepared VM contains a Linux command line scavenger hunt
  activity, available here:
  \url{https://github.com/tsani/scavenger-hunt}
\end{itemize}

\section*{Instructional sequence}

\begin{itemize}
\item[5 mins.]
  Facilitators will greet the participants and present the recap and lesson
  agenda.
\item[20 mins.]
  Two things happen concurrently during this period:
  \begin{itemize}
  \item
    A few participants will be pulled aside by one facilitator and will discuss
    the boot sequence of a computer following a video:
    \url{https://www.youtube.com/watch?v=zyHoBzm5taw}
  \item
    The remaining participants will reboot a computer multiple times and take
    note of visible and audible changes in the computer, in order to try to
    empirically figure out what the different stages of the boot sequence are.
    These participants will construct a timeline of events that take place
    during the boot sequence.
  \end{itemize}
\item[10 mins.]
  The participants that were taken aside will work the participants who observed
  the boot sequence to bridge their knowledge. They will work together to
  annotate the timeline with the different stages of the boot sequence outlined
  in the slides.
\item[5 mins.]
  Facilitators will show a Crash Course Computer Science video on operating
  systems.
\item[5 mins.]
  Facilitators will introduce Linux by distinguishing distributions from kernels
  from other types of Linux OSes such as Android.
  Then, participants will boot up their virtual machines and log in.
  Facilitators will briefly touch on the \texttt{cd} and \texttt{ls} commands to
  prepare participants to do the scavenger hunt activity.
\item[25 mins.]
  Participants will complete the scavenger hunt activity as facilitators
  circulate to assist them.
\item[5 mins.]
  Facilitors will briefly recap the lesson and mention the topics to be
  discussed next week.
\end{itemize}

\end{document}
