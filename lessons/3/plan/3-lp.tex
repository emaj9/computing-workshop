\documentclass[11pt]{article}

\usepackage[margin=2cm]{geometry}
\usepackage{hyperref}

\title{3 -- Types, pattern matching, and the ALU}
\author{Eric Mayhew \& Jacob Errington}
\date{}
\newcommand{\cwurl}{https://www.computing-workshop.com/}
\newcommand{\cwpdf}{\cwurl pdf/}

\newcommand{\codeworld}{\href{http://code.world/}{code world}}

\begin{document}

\maketitle

In this lesson, participants are introduced to two key concepts of functional
programming: pattern matching and recursion.

\section*{Overview}

\subsection*{Learning objectives}

By the end of the lesson, students will be able to:
%
\begin{itemize}
\item Write recursive functions that process lists using pattern matching.
\item Write interactive programs, with \codeworld.
\item Explain how circuitry can perform arithmetic, by building a half-adder
  using logic gate ICs.
\end{itemize}

\subsection*{Materials}

To run this lesson, the folloing materials are necessary:

\begin{itemize}
\item The lesson slides, \url{\cwpdf{3-presentation.pdf}}.
\item One breadbaord for each group and enough of the following electronic
  components:
  NPN transistors,
  100~$\Omega$ resistors,
  9~V batteries,
  5~V voltage regulators,
  jumper wires (male to male),
  assorted LEDs,
  integrated circuits from the 7400-series (7408 (quad 2-input AND gate), 7486
  (quad 2-input XOR gate)).
\item Each participant needs a computer to work on, with access to the web.
\end{itemize}

\section*{Instructional sequence}

\begin{itemize}
\item[5 mins.]
  Facilitators will greet the participants and present the lesson agenda.
\end{itemize}

\section*{Homework}

\end{document}
