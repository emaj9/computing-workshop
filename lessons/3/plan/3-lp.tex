\documentclass[11pt]{article}

\title{3 -- Pattern matching and ALU}
\author{Eric Mayhew \& Jacob Errington}
\date{}

\newcommand{\cwurl}{https://www.computing-workshop.com/}
\newcommand{\cwpdf}{\cwurl pdf/}
\newcommand{\bootstrapurl}{\url{\cwurl hs/getting-started.hs}}

\newcommand{\codeworld}{\href{http://code.world/}{code world}}

\begin{document}

\maketitle

In this lesson, participants are introduced to two key concepts of functional
programming: pattern matching and recursion.

\section*{Overview}

\subsection*{Learning objectives}

By the end of the lesson, students will be able to:
%
\begin{itemize}
\item Write recursive functions that process lists using pattern matching.
\item Write interactive programs, with \codeworld.
\item Explain how circuitry can perform arithmetic, by building a half-adder
  using logic gate ICs.
\end{itemize}

\subsection*{Materials}

To run this lesson, the folloing materials are necessary:

\begin{itemize}
\item The lesson slides, \url{\cwpdf{3-presentation.pdf}}.
\item One breadbaord for each group and enough of the following electronic components:
  NPN transistors,
  100~$\Omega$ resistors,
  9~V batteries,
  5~V voltage regulators,
  jumper wires (male to male),
  assorted LEDs,
  integrated circuits from the 7400-series (7408 (quad 2-input AND gate), 7486
  (quad 2-input XOR gate)).
\item Each participant needs a computer to work on, with access to the web.
\item A basic Haskell file for getting started with interactive programs in
  codeworld: \bootstrapurl.
\end{itemize}

\section*{Instructional sequence}

\begin{itemize}
\item[5 mins.]
  Facilitators will greet the participants and present the lesson agenda.
\item[20 mins.]
  Using the slides, facilitators will recall types in Haskell, paying special
  attention to function types. This leads into an explanation of pattern
  matching, by walking through the stoplight example.
  Facilitators will follow-up with a discussion of recursion, by defining the
  \texttt{length} function for lists recursively, and walkthrough through its
  evaluation.
\item[20 mins.]
  With this knowledge of pattern matching, facilitators will walk students
  through a concrete example of pattern matching on \codeworld{} by creating a
  program that displays a shape whose colour is chosen by cycling through a list
  when a key is pressed.

  Students will get started by pasting a premade bootstrap file \bootstrapurl{}
  into \codeworld. Then they will follow along with the projected facilitator's
  screen.
\end{itemize}

\end{document}
