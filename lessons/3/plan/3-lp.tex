\documentclass[11pt]{article}

\usepackage[margin=2cm]{geometry}
\usepackage{hyperref}

\title{3 -- Types, pattern matching, and the ALU}
\author{Eric Mayhew \& Jacob Errington}
\date{}
\newcommand{\cwurl}{https://www.computing-workshop.com/}
\newcommand{\cwpdf}{\cwurl pdf/}

\newcommand{\codeworld}{\href{http://code.world/}{code world}}

\begin{document}

\maketitle

In this lesson, participants are introduced to two key concepts of functional
programming: pattern matching and recursion. By reimplementing functions for the
standard library, participants will focus on the
syntax for defining functions in Haskell. This will include recursion and
pattern matching, the only control flow mechanism in Haskell. Finally, this
lesson will wrap up hardware by covering the Arithmetic and Logic Unit, as
participants create a half adder on their breadboard.

\section*{Overview}

\subsection*{Learning objectives}

By the end of the lesson, students will be able to:
%
\begin{itemize}
\item Read Haskell type signatures, identifying: function inputs and outputs,
  type variables, class constrants.
\item Write Haskell functions using pattern matching on inputs.
\item Reimplement several functions from the standard library, including
  recursive functions.
\item Demonstrate how circuitry can perform arithmetic by building a half-adder
  using logic gate ICs.
\end{itemize}

\subsection*{Materials}

To run this lesson, the folloing materials are necessary:

\begin{itemize}
\item The lesson slides, \url{\cwpdf{3-presentation.pdf}}.
\item One breadbaord for each group and enough of the following electronic
  components:
  NPN transistors,
  100~$\Omega$ resistors,
  9~V batteries,
  5~V voltage regulators,
  jumper wires (male to male),
  assorted LEDs,
  integrated circuits from the 7400-series (7408 (quad 2-input AND gate), 7486
  (quad 2-input XOR gate)).
\item Each participant needs a computer to work on, with access to the web.
\end{itemize}

\section*{Instructional sequence}

\begin{itemize}
\item
Insturctional sequence is outlined in the slides.
\end{itemize}

\section*{Homework}

\end{document}
