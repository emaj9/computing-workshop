\documentclass[11pt]{article}

\usepackage[margin=2.0cm]{geometry}
\usepackage{hyperref}

\newcommand{\code}{\texttt}

\author{%
  Eric Mayhew \& Jacob Errington%
  \footnote{Special thanks to McGill's Office of Student Life and Learning}
}
\title{Syllabus}
\date{Winter 2018}

\begin{document}

\maketitle

\begin{description}
  \item[Website:]
    \url{http://computing-workshop.com/}

  \item[Location:]
    B21, 651 rue Sherbrooke Ouest
    (Northeast corner of University street)

  \item[Time:]
    From 1:30~PM to 3:00~PM on the following days:
    \begin{enumerate}
      \item January 16,
      \item January 23,
      \item January 30,
      \item February 6,
      \item February 13,
      \item February 20,
      \item February 27,
      \item March 6,
      \item March 13,
      \item March 20,
      \item March 27.
    \end{enumerate}

    After each session there is a one-hour homework tutorial during which the
    facilitators will give hands-on help with whatever homework was assigned
    during the session.
\end{description}

\section*{Description}

Computing Workshop is a 9 session workshop designed to help people understand
how computers and websites work. This workshop will cover hardware, software,
and web development to individuals with little to no background in computers
using cooperative, problem-based learning methods. Each class is 1.5 hours long
and has an estimated 1 hour worth of homework that accompanies each lesson.

\section*{Rationale}

We created this workshop to provide people with the support to feel comfortable
using a computer. Technology is pervasive in our society, but rarely are
individuals provided with meaningful education in how to use it. Our workshop
looks to activate participant's motivation by providing them with hands on,
interactive lessons covering fundamental knowledge with respect to computers.
This workshop uses a co-constructivist teaching approach to help our
participants develop their own tools and understanding of computers.

\section{Lesson sequence}

\begin{enumerate}
    \setcounter{enumi}{0}
  \item Intro, hardware, abstraction
  \item Websites, boot sequence, programming
  \item ...
  % \item What happens when you turn a computer on? Basic Haskell types, intro to HTML
  % \item Operating Systems, Type classes/kind, Intro to CSS
  % \item Applications, Functions, Intro to Java
  % \item Programs, Higher order functions, Intro to Hakyll
  % \item Internet, Functors, website aethestics
  % \item Project management, Monads, Website hackathon
  % \item Life long learners in the online community, hackathon
  % \item Conclusion
\end{enumerate}

\section*{Learning goals}

\subsection*{Computers}

\begin{itemize}
  \item identify key hardware components and their function;
  \item identify core services provided by an operating system;
  \item identify the boundaries and relationships between application,
    operating system, and hardware;
  \item explain the process of downloading a resource from a remote host;
  \item explain the role of HTML, CSS, and JavaScript in displaying a web page
    in a browser;
  \item categorize different networking protocols (application vs. hardware);
  \item break down the process from computer power-on to operating system
    startup;
  \item interact with online communities and tools;
  \item appreciate open source material;
  \item navigate online resources and take autonomy of their own learning
\end{itemize}

\subsection*{Programming}

\begin{itemize}
  \item Translate a word problem into code: identify key types and dataflow;
  \item use Haskell libraries effectively;
  \item differentiate between values, types, types classes, and kinds;
  \item type check expressions;
  \item implement and use higher-order functions and partial applications;
  \item use the \code{IO} monad to create programs with side-effects;
  \item understand general patterns (e.g. \code{Functor}) when applicable and
    use them;
  \item
    use \code{do}-notation for a variety of monads:
    \code{Maybe}, \code{Either a}, \textellipsis;
\end{itemize}

\subsection*{Web development}

\begin{itemize}
  \item Design and implement a website;
  \item characterize the historic trends in web development;
  \item apply appropriate aesthetics to the design of websites:
    golden ratio, negative space, color theory, \textellipsis;
  \item use appropriate elements of HTML (hamburger text markup language);
  \item make responsive websites;
  \item build reusable web components;
  \item use JavaScript to create dynamic websites;
  \item use appropriate web development tooling (e.g. static site generators).
\end{itemize}

\end{document}
