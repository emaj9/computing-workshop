\documentclass[11pt]{article}

\usepackage[margin=2.0cm]{geometry}
\usepackage{hyperref}

\newcommand{\code}{\texttt}

\author{%
  Eric Mayhew \& Jacob Errington%
  \footnote{%
    Special thanks to Building21 (McGill's Office of Student Life and Learning)
    and to Anita Parmar.
    Computing Workshop would not have been possible without her tremendous
    support!
  }
}
\title{Computing Workshop: Hardware Syllabus}
\date{Fall 2018}

\begin{document}

\maketitle

\begin{description}
  \item[Website:]
    \url{http://computing-workshop.com/}

  \item[Location:]
    B21, 651 rue Sherbrooke Ouest
    (Northeast corner of University street)

  \item[Time:]
    Options: Wed at 4 - 6, Thursday from 3 - 5
    \begin{itemize}
      \item September something,
    \end{itemize}

  \item[Goal:]
    At the end of the workshop, particpants will be able to identify the core
    hardware components of a modern computer and describe their function.
    They will also be able to construct simple hardware components using discrete
    circuitry.
    They will be able to describe how these components work together to form
    a computer.
\end{description}

\section*{Description}

Computing Workshop: Hardware Unit is the half of Computing Workshop that focuses
on the physical components of computers. This unit touches on three essential
parts of a computer: memory, processor, and arithmetic logic unit.
Divided into six two-hour sessions, participants will understand the
functionality of these components by recreating them using breadboards and
circuitry.
This workshop will also explore how these components operate jointly to make up
a computer as we know it.

\section*{Rationale}

We created this workshop to provide people with the support to feel comfortable
using a computer. Technology is pervasive in our society, but rarely are
individuals provided with meaningful education about how to they work. Our workshop
looks to activate a participant's motivation by providing them with hands on,
interactive lessons covering fundamental knowledge with respect to computers.
This workshop uses a co-constructivist teaching approach to help our
participants develop their own understanding of computers.

\vfill
\section*{Lesson sequence}

\begin{enumerate}
    \setcounter{enumi}{-1}
  \item Intro, first look at hardware
  \item Circuits and breadboards: the tools of our exploration
  \item Intergrated circuits (ICs), the half adder, and ALU
  \item More ICs, memory, and set-reset latch
  \item Intro to operating systems (OSs), and the boot sequence
  \item Deeper delve into OSs, graphics, and conclusion
\end{enumerate}

\section*{Learning goals}

\subsection*{Computers}

\begin{itemize}
  \item identify key hardware components of a computer;
  \item describe each hardware component's function;
  \item effectively use the electronic components necessary in creating circuits;
  \item recreate basic hardware components using circuits;
  \item identify core services provided by an operating system;
  \item break down the boot process, from computer power-on to user login.
\end{itemize}

\end{document}
