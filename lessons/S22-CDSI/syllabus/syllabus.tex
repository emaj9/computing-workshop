\documentclass[11pt]{article}

\usepackage[margin=2.0cm]{geometry}
\usepackage{hyperref}
\usepackage{multicol}
\usepackage{listings}

\lstset{
  basicstyle=\ttfamily
}

\author{%
  Jacob Errington \& Eric Mayhew%
  \footnote{%
  Funding and venue is kindly provided by McGill's Computational and Data Systems Initiative (CDSI), \url{https://mcgill.ca/cdsi}
  }
}
\title{\vspace{-2em}Computing Workshop: Intro Python -- Syllabus}
\date{Summer 2022}

\begin{document}

\maketitle

\begin{multicols}{2}
  \begin{description}
    \item[Website]
      \url{https://computing-workshop.com/}
    \item[Location]
      Burnside Hall, 5th floor, Geographic Information Center (GIC)
    \item[Time]
      15 - 19 August, 10~am - 3~pm with lunchbreak noon - 1~pm
    \item[Audience]
      Incoming graduate students without background in computer science,
      interested in using Python for their research
  \end{description}

  \subsection*{Learning Goals}

  At the end of the workshop, a participant will be able to:

  \begin{itemize}
  \item Use control flow mechanisms, e.g. \lstinline!if!- and \lstinline!for!-statements.
  \item Organize a program into functions; follow the Single Responsibility Principle.
  \item Use common I/O mechanisms: standard input/output, files, HTTP.
  \item Run Python scripts on their computer.
  \item Justify/critique the use of software in solving real-world problems.
  \end{itemize}

  \subsection*{Lesson sequence}

  \begin{description}
  \item[15 Aug.] Welcome! Land acknowledgement. Statements, expressions, types,
    and values. Program execution. Compilation vs interpretation.
  \item[16 Aug.] Control flow: conditions, loops, functions.
  \item[17 Aug.] Data structures: lists, dictionaries, trees, graphs.
  \item[18 Aug.] The environment: filesystem, network, standard I/O. Batch vs interactive.
  \item[19 Aug.] Applications: basic data science with NumPy, SciPy.
  \end{description}

  \columnbreak
  \section*{Description}

  Software is created by writing a program in a \emph{programming language.} This course
  provides a guided first step towards software development by equipping you
  with fundamental skills in a very popular and beginner-friendly programming
  language, \emph{Python.} These fundamentals are easily transferrable to other
  programming languages, too!

  This is a five-session, one-week intensive workshop, consisting of twenty total
  hours of classtime. Each day will consist of a 2-hour morning session and a
  2-hour afternoon session. The morning session will lean more on the
  theoretical side, with lectures punctuated by small group discussions and
  worksheets. The afternoon session will lean more on the practical side, where
  you will work on coding problems in pairs to cement your understanding of the
  morning material.

  This course aims to give a foundation for further study of popular kinds of
  Python software development, in particular machine learning and data science.
  At the same time, we aim to impart to you a critical understanding of software
  development, and especially of machine learning: you must think not only of
  whether software \emph{can} solve a problem, but especially of whether it
  \emph{should.} Expect to make ethical considerations before and alongside the
  software development you will undertake in solving problems in this course!
\end{multicols}

\end{document}
