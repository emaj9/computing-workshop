\documentclass[11pt]{article}
\usepackage{listings}

\author{Computing Workshop}
\title{Co-teaching Python Review - Version A}
\date{Summer 2022}

\usepackage[margin=2.0cm]{geometry}

\begin{document}

\maketitle

Review the following concepts and make sure you know them well enough to explain them to someone else. Read the
following document and answer the embedded questions to assess your understanding. If something is unclear, please don't hesitate to ask for an explination.

\section*{Types and values}

Expressions evaluate down to a value. Every value has a \textbf{type}. The four primitive types are \texttt{int, float, string,} and
\texttt{bool}. In the following table, write 2 possible values for each type.
\\

\def\arraystretch{2}
\begin{tabular}{| c | p{34em}|} \hline
  Type & Value \\ \hline
  int & ~ \\ \hline
  float & ~ \\ \hline
  string & ~ \\ \hline
  bool & ~ \\ \hline
\end{tabular}
\\

Types allow us to control or restrict what we can do to data. For example, Try to subtract an \texttt{int} from a
\texttt{string} in the interpreter. What does the interpreter tell you and why is it saying that?

\section*{Variables}

Variables are pointers to a value. Here is what a variable looks like in Python \texttt{time = ``2pm''}. The \texttt{=}
operator binds a variable name on the left (in the example \texttt{time}) to value on the right (in this example, the
string \texttt{2pm}). Now, to the computer, \texttt{time} is \emph{synonymous} with \texttt{2pm}. You can update the
value of a variable by reassigning it, like this \texttt{time = ``2:10pm''}.

Consider the following snippet of code:
\begin{lstlisting}
  a = 4
  b = a + 1
  a = a + "two"
  print(a)
  print(b)
\end{lstlisting}

What will the program print for \texttt{a} and \texttt{b}? Run the code in your interpreter to check your answer!

\end{document}
