\documentclass[11pt]{article}
\usepackage{listings}

\author{Computing Workshop}
\title{Co-teaching Python Review - Version B}
\date{Summer 2022}

\usepackage[margin=2.0cm]{geometry}

\begin{document}

\maketitle

Review the following concepts and make sure you know them well enough to explain them to someone else. Read the
following document and answer the embedded questions to assess your understanding. If something is unclear, please don't hesitate to ask for an explination.

\section*{Data structures}

\emph{Data structures} are a piece of data that holds more data in it.
Lists, for example, are the most common data structure in programming. The name of a list is on the left hand side of the \texttt{=}
and the contents are on the right, surrounded by square brakets and each element of the list is comma separated. Consider the following list:

\begin{lstlisting}
  prime_colours = [``red'', ``yellow'', ``blue'']
\end{lstlisting}
~

You can access an element of a list through its \emph{index}. The index is the order the list is in, from left to right,
but be careful as computers start counting at zero! Here is how we would get the first index of the \texttt{prime\_colours}
list: \texttt{prime\_colour[0]}.

Test your understanding by coding your own list with at least 3 big numbers (at least 7 digits each) in it using repl.it. Once you have successfully
coded your list, sum the number of the first and last element of the list by accessing it through the list indexs.


\end{document}
