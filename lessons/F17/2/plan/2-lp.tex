\documentclass[11pt]{article}

\usepackage[margin=2cm]{geometry}
\usepackage{hyperref}

\title{2 -- Integrated circuits and memory}
\author{Eric Mayhew \& Jacob Errington}
\date{}
\newcommand{\cwurl}{https://www.computing-workshop.com/}
\newcommand{\cwpdf}{\cwurl pdf/}

\newcommand{\codeworld}{\href{http://code.world/}{code world}}

\begin{document}

\maketitle

This lesson guides its participants up the ladder of abstraction from
transistors to integrated circuits, which bundle up useful logical operations
into their own electronic component. The development of integrated circuits
allowed computers to be miniaturized. Participants will use integrated circuits
for AND and OR gates in order to build an S-R latch, which will illustrate the
operation of memory. Finally, participants will use \codeworld{} to explore
Haskell functions and types.

\section*{Overview}

\subsection*{Learning objectives}

By the end of this lesson, students will be able to:
%
\begin{itemize}
\item Physically construct a NOT gate using a discrete transistor on a breadboard.
\item Explain the operation of computer memory by physically constructing an S-R
  latch using integrated circuits.
\item Use Haskell functions to draw simple images.
\item Interpret simple Haskell type signatures.
\item Differentiate between definitions and expressions.
\end{itemize}

\subsection*{Materials}

To run this lesson, the following materials are necessary:

\begin{itemize}
\item The lesson slides, \cwpdf{2-presentation.pdf}.
\item One breadboard for each group and enough of the following electronic
  components:
  NPN transistors,
  100~$\Omega$ resistors,
  9~V batteries,
  5~V voltage regulators,
  jumper wires (male to male),
  assorted LEDs,
  integrated circuits from the 7400-series (7408 (quad 2-input AND gate), 7432
  (quad 2-input OR gate)).
\end{itemize}

\section*{Instructional sequence}

\begin{itemize}
\item[5 mins.]
  Facilitators will greet the participants and present the lesson agenda.
\item[10 mins.]
  Following a diagram in the slides, participants will build a NOT gate on their
  breadboard using transistors.
\item[20 mins.]
  Facilitators will introduce integrated circuits as packaged-up
  transistor-based circuits.
  To get comfortable with ICs, participants will test the AND and OR gates by
  manipulating their inputs and observing with LEDs their outputs.
  Diagrams of the test circuits are provided in the slides.
\item[30 mins.]
  When participants are comfortable with ICs, the facilitators will introduce
  computer memory and its operation by showing a video clip linked in the
  slides. (1:27 to 3:12.)
  %
  Participants will construct an S-R latch using ICs following a diagram in the
  slides.
\item[20 mins.]
  Once students complete the S-R latch exercise, facilitators will introduce
  functions as machines with one input and one output.
  Facilitators will explain how to read and interpret a Haskell type signature,
  paying close attention to concrete types, tuple types, and function types.
  To demonstrate the use of functions, facilitators will program a picture of a
  village. Participants will follow along with this code on their computers.
  Once this code is complete, participants will be invited to extend (or
  completely change) it in small groups or individually.
\end{itemize}

\section*{Homework}
For homework, participants will create their own picture using \codeworld.
\end{document}
