\documentclass{exam}

\title{Breadboard reverse engineering}
\author{}
\date{}

\usepackage{float}
\usepackage{tikz}

\newcommand{\answerbox}[1]{%
  \begin{figure}[H]
    \begin{tikzpicture}
      \draw
      (0,0) rectangle (\textwidth, #1\textheight);
    \end{tikzpicture}
  \end{figure}
}

\begin{document}

\maketitle

Your group will be assigned a logical circuit to implement on a breadboard.

\begin{questions}
  \question
  Draw up the truth table for the circuit you have been assigned.

  \answerbox{0.20}

  \question
  Draw a complete diagram of the circuitry involved in implementing the logical
  circuit.
  This diagram should include all power sources, transistors, LEDs, and
  integrated circuits that will go on the breadboard.

  \answerbox{0.30}

  \question
  Implement on your breadboard the design that you have created.

  When you are finished, trade breadboards with another team.

  \vfill
  \question
  Reverse-engineer the breadboard you obtained from the other group by
  drawing its complete circuit diagram.

  \answerbox{0.50}

  When you are finished, discuss with the team you traded breadboards with and
  compare your diagram with theirs.
\end{questions}

\end{document}
