\documentclass[11pt]{article}

\usepackage[margin=2cm]{geometry}
\usepackage{hyperref}

\title{5 - Datatypes and higher-order functions}
\author{Jacob Errington \& Eric Mayhew}
\date{}

\newcommand{\cwurl}{https://www.computing-workshop.com/}
\newcommand{\cwpdf}{\cwurl pdf/}

\begin{document}

\maketitle

In this lesson, participants will explore higher-order functions by
implementing a few common ones and about datatypes by creating their own to
represent a person.

\section*{Overview}

\subsection*{Learning objectives}

By the end of the lesson, students will be able to:

\begin{itemize}
\item
  Implement the following higher-order functions: \texttt{filter} and
  \texttt{map}.
\item
  Define new datatypes using the \texttt{data} keyword and implement functions
  operating on these datatypes.
\item
  Explain the difference between data constructors and plain functions.
\item
  Name the key ideas that underpin the \texttt{Functor} typeclass.
\end{itemize}

\subsection*{Materials}

To run this lesson, the following materials are necessary:

\begin{itemize}
\item
  The slides, available here:
  \url{\cwpdf 5-Datatypes-and-higher-order-functions.pdf}
\item
  Each participant requires a computer with VirtualBox installed, and each
  participant needs to download and unzip the prepared virtual machine:
  \url{https://files.jerrington.me/arch.zip}.
  (This is the same virtual machine as in the previous lesson.)
\end{itemize}

\section*{Instructional sequence}

The instructional sequence for this lesson is outlined in the slides.

\end{document}
