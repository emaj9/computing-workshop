\documentclass[11pt]{article}

\usepackage[margin=2cm]{geometry}
\usepackage{hyperref}

\title{7 -- CSS}
\author{Eric Mayhew \& Jacob Errington}
\date{}
\newcommand{\cwurl}{https://www.computing-workshop.com/}
\newcommand{\cwpdf}{\cwurl pdf/}

\begin{document}

\maketitle

Now that participants have an understanding of how to create \emph{content} in
HTML, they will learn how to \emph{style} their pages using CSS.

\section*{Overview}

\subsection*{Learning objectives}

By the end of this lesson, students will be able to:
%
\begin{itemize}
\item Explain syntactic elements of CSS: selectors and tags.
\item Link CSS stylesheets with HTML documents.
\item Use CSS to control various style properties (foreground/background text
  color, font size, margins, padding) of HTML elements.
\end{itemize}

\subsection*{Materials}

To run this lesson, the following materials are necessary:

\begin{itemize}
\item The lesson slides, \cwpdf{7-css.pdf}.
\item Each participant needs to bring their computer, and needs a text editor
  installed on it.
\end{itemize}

\section*{Instructional sequence}

Refer to the slides.

\section*{Homework}
\end{document}
